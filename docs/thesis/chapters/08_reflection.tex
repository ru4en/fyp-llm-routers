\chapter{Reflection}
\label{ch:reflection}


\subsection{Future improvements to the Router Library}
\label{sec:results-future-improvements-to-the-router-library}

With the completion of this project, I have gained valuable insights into the challenges and opportunities in the field of routing tasks using NLI models. Learning about the intricacies of NLI models as well as the pipelines from Hugging Face Library has been a rewarding experience. The Router library is a promising tool for routing tasks, although it is still in its early stages of development It has helped me gain a better understanding working with not only NLI models but a better understanding of LLM pipelines in general. Researching and implementing the Router library has been a valuable learning experience, and I have gained a deeper understanding of the challenges and opportunities in this field. Building the Router library has also helped me get better acquainted with deploying python libraries and setting up a proper CI/CD pipeline.

Some of the key takeaways I could improve upon in the future are:
\begin{itemize}
    \item Improve my knowledge on multithreading and parallel processing.
    \item Get a better understanding of fine tuning models and how to do it effectively. Since I failed to get the model to work as well as the base model.
    \item Improve my knowledge of the Hugging Face Transformers library and how to use the other models available in the library.
\end{itemize}


As discussed in the previous section, there are several areas for improvement and questions that need to be addressed. Further  work is needed to improve the reliability of the Router library.
However, I am confident enough to use the tool router plugins generated for the OpenWebUI project in my self hosted instance of it and looking forward for a fix of the API to be able to use the agent router as well. All in all, I am pleased with the results and the progress I have made in this project.

