\chapter{Conclusions and Future Work}
\label{ch:con}


\section{Conclusion}
\label{sec:results-conclusion}
In conclusion, the initial results demonstrate that Router using NLI models can be effective in selecting the appropriate agent or tool for a given task with a high degree of accuracy and retevelely low latency. The tool router outperformed the agent router, indicating that more work is needed to improve the agent router's performance.  Over the benchmark made using the 70 synthetic prompt generation, the tool router recorded 45 passes, 18 warnings and only 7 fails, whereas the agent router managed just 28 passes, 13 warnings and 29 fails. This demonstrates that the entailment approach was much more reliable for selecting the correct tool, but the agent router was less effective. 

The results also highlight the importance of providing sufficient context to the router. The tool router was able to make more accurate decisions when the prompts provided clear hints about the tool to be used.

The routers were build around facebooks BART-Large-MNLI model, which was last updated in 2023 (as of writing). More recent models such that are potentially based on more advanced architectures such as the deepseek model with more specific training data could be used to improve the performance of the router. Although the results are promising, there are several improvements and questions that need to be addressed. I will go into more detail about these in the next section.

Although my router library is not yet ready for production use, and this research is more of a proof of concept, it does demonstrate the potential of using NLI models for routing tasks, Furthermore, this project has given me valuable insights into the challenges and opportunities in this field and has made me more aware of the limitations of current routing mechanisms. The research has also highlighted the need for further investigation into the use of NLI models for routing tasks, particularly in production environments. This is an area that I am keen to explore further in future work or building apon the work done in this project.


\section{Recommendations for Future Work}
\label{sec:results-recommendations}

Building on these results, several promising directions for future research emerge:

\begin{itemize}
    \item \textbf{User-Centric Evaluation:} Expanding the evaluation to include real user studies would provide deeper insights into the router's practical effectiveness. Collecting authentic user queries and systematically tracking router decisions could help identify blind spots and areas for improvement. Additionally, integrating analytics would enable continuous monitoring of router accuracy and facilitate data-driven refinements over time.
    
    \item \textbf{Model Enhancements:} Improving the router model itself offers significant potential. Possible strategies include: (a) applying few-shot or reinforcement learning to enable the router to adapt its decision-making dynamically, (b) implementing hierarchical routing, where a lightweight intent classifier initially filters queries and only escalates to the LLM-based router when necessary, and (c) developing ensemble approaches, such as combining GPT-4 with a smaller local model for faster, resource-efficient decisions. Following best practices—starting with simple solutions and iteratively evolving based on performance metrics—could yield a more robust and adaptable system.
    
    \item \textbf{Fine-Tuning for Agent Routing:} Using a fine tuned model specifically for the agent router could substantially enhance its performance. This would involve training on a dedicated dataset of routing tasks, enabling the model to learn optimal agent selection for diverse scenarios. While this approach requires considerable data and computational resources, it holds promise for significantly improving agent router accuracy and reliability.
\end{itemize}

Pursuing these avenues could lead to a more comprehensive understanding of routing strategies and further advance the state of the art in automated tool and agent selection.

