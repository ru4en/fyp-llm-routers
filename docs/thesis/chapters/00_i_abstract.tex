\chapter*{\center \Large  Abstract}

\noindent In the current landscape of large language models, users are confronted with a plethora of models and tools each offering a unique blend of specialisation and generality. This project proposes the development of a dynamic middleware \textit{router} designed to automatically assign user queries to the most appropriate model or tool within a multi agent system. By using zero shot Natural Language Inference models, the router will evaluate incoming prompts against criteria such as task specificity and computational efficiency, and \textit{route} the prompt to the most effective model and/or allow specific tools relevant that the model could use.

The proposed framework is underpinned by three core routing mechanisms:

\begin{itemize}
    \item Firstly, it will direct queries to cost effective yet sufficiently capable models, a concept that builds on existing work in semantic routing \cite{ong2025routellmlearningroutellms}.
    
    \item Secondly, it incorporates a tool routing system that automatic invocation of specialised functions, thus streamlining user interaction and reducing inefficiencies currently inherent in systems like OpenAI's and Open Web UI. Furthermore this could also reduce inefficiencies in the recent reasoning models addressing the observed dichotomy between underthinking with complex prompts and overthinking with simpler queries when reasoning is manually toggled which can be costly and could cause hallucination.
    
    \item Thirdly, while the primary focus remains on model and tool routing, this work will preliminarily explore the potential application of the routing architecture as a security mechanism. Initial investigations will examine the theoretical feasibility of leveraging the router's natural language understanding capabilities to identify adversarial prompts. This includes a preliminary assessment of detection capabilities for prompt engineering attempts, potential jailbreaking patterns, and anomalous tool usage requests. However, given the rapidly evolving nature of LLM security threats and the complexity of implementing robust safeguards, comprehensive security features remain outside the core scope of this research. Although this aspect represents a promising direction for potential future work, particularly as the field of LLM security continues to grow.
\end{itemize}

\noindent By integrating these mechanisms, the research aims to pioneer a more efficient, modular, and secure distributed AI architecture. This architecture not only optimises resource allocation but also reinforces system integrity against emerging adversarial threats, thereby contributing novel insights into the development of next generation LLM deployment strategies.