\chapter{Implementation}
\label{ch:implementation}

\section{Router Development}

\subsection{Generic Prompt to Topic Router (Prototype)}
\label{sec:generic_router_dev}

Following model selection in the research phase, I developed a prototype router capable of classifying incoming prompts into predefined topic categories. This prototype served as the foundation for subsequent development efforts and allowed us to establish baseline performance metrics. The router is accessible via the \texttt{llm\_routers} library as the \texttt{Router()} class.

The fundamental design takes a dictionary style hash map within an array structure to define topics and their descriptions:

\begin{lstlisting}[language=Python, caption={Example Router Usage}, breaklines=true]

TOPICS = [
    {"NEWS": "Breaking stories, current events, and latest headlines from around the world, updated in real time."},
    {"Entertainment": "Latest updates on movies, music, celebrities, TV shows, and pop culture highlights."},
    {"Sports": "Live scores, match results, player updates, and coverage of major sporting events worldwide."}
]

topic_router = Router(TOPICS)

agent_router.route_query("Dr Who display extends hours to attract visitors")

agent_router

# >> ("Entertainment", 0.73494)
\end{lstlisting}





\subsection{Python Library
(Development)}\label{python library development}

% ### TODO: ADD MORE IN THIS SECTION



The next phase involved creating a Python library that extends the generic router and encapsulate the routing functionality. This library is designed to be modular, extensible, and user friendly, allowing developers to easily integrate
it into their existing systems. The library is structured to support
multiple routing mechanisms, including the basic prompt to topic router,
agent selection router, and tool selection router.



\begin{figure}[H]
    \centering
    \includegraphics[width=1.0\textwidth]{figures/UML.png}
    \caption{UML Diagram of the Router Library}
    \label{fig:uml_diagram}
\end{figure}




\begin{verbatim}
pip install git+https://github.com/ru4en/llm_routers.git
\end{verbatim}


\subsection{Library Architecture Design Principles}

The library architecture implements the following design principles:

\begin{enumerate}
    \item \textbf{Modular Component Structure}: The system is organised into three main components:
    \begin{itemize}
        \item \texttt{Router}: The core classification engine that maps prompts to predefined topics with confidence scores.
        \item \texttt{AgentRouter}: A router that selects appropriate AI agents based on query requirements.
        \item \texttt{ToolRouter}: A router that matches queries to optimal tools for task completion.
    \end{itemize}

    \item \textbf{Consistent API Design}: All router types implement two primary methods:
    \begin{itemize}
        \item \texttt{route\_query(user\_prompt)}: Routes a single prompt to the appropriate topic/agent/tool.
        \item \texttt{batch\_route([user\_prompt, user\_prompt])}: Efficiently processes multiple prompts in a single operation.
    \end{itemize}

    \item \textbf{Dynamic Configuration}: Each router provides candidate management functions:
    \begin{itemize}
        \item \texttt{add\_candidate(name, description)}: Dynamically expands the routing options.
        \item \texttt{remove\_candidate(name)}: Removes options from the router's consideration set.
    \end{itemize}

    \item \textbf{Performance Optimisations}: Several enhancements improve real world deployment characteristics:
    \begin{itemize}
        \item Batched processing for efficient handling of multiple requests.
        \item Asynchronous operation options via \texttt{async route\_query()}.
    \end{itemize}
\end{enumerate}

The implementation includes comprehensive error handling and detailed logging to support easy integration into existing pipelines.


\subsection{Evaluation Framework Creation}
\label{evaluation framework creation}
To facilitate adequate testing and performance measurement, I developed a simple testing and demo scripts that would use a set of data to enables systematic assessment of routing accuracy and computational efficiency across diverse scenarios whilst also allowing to try out the library.


\subsubsection{Testing Script}
\label{testing script}
A test script that tested against a set of synthetically generated prompts was created. This simple script used the \texttt{llm\_routers} as one of its dependencies and, for every single prompt in the synthetic dataset, invoked the router twice to predict:

\begin{enumerate}
    \item The best agent to use.
    \item The top 3 tools for the job.
\end{enumerate}

\begin{verbatim}
python3 src/test/test.py
\end{verbatim}


\subsubsection{Demo CLI Script}
\label{demo cli script}
An extra script was also created to lets the user interact with routers allowing them to input a prompt and predict a set agent and tools for the given prompt. The tools and agents from the options stays the same from the synthetic dataset although the user get to input the prompt they want.

\begin{verbatim}
python3 src/test/demo.py
\end{verbatim}


\subsection{Synthetic Dataset Generation}
\label{synthetic dataset generation}
To validate the routers, I constructed synthetic test datasets. In \texttt{src/test/syn\_data/data.py}, I defined a set of agents (e.g., "Adam" the developer, "Eve" the designer) and tools with descriptions, along with a list of test queries mapping to expected agents and tools. For example, the query \textit{"Write a Python application to track stock prices"} is labelled with agent "Adam" and tools \texttt{["IDE", "Terminal"]}. This synthetic data covers a range of domains such as coding tasks, design tasks, and research tasks. During testing, the router modules are run against these cases to measure correctness (see \texttt{src/test/test.py}). This approach allows systematic evaluation without needing large real world logs. Moreover, generating synthetic queries ensures controlled coverage of edge cases (such as queries that should route to multiple tools or ambiguous cases).

\begin{quote}
\textbf{NOTE:} Synthetic Dataset, which is only used to demonstrate the feasibility of this module, was generated using Chat GPT.
\end{quote}

\begin{lstlisting}[language=Python, caption={Example of the synthetic dataset}, breaklines=true]
agents = {
    "Adam": "Coder and Developer Agent - Specialises in Python and JavaScript development; creates scripts and applications.",
    "Eve": "Designer and Artist Agent - Expert in UI/UX and Graphics Design; produces content.",
    ...
}
tools = {
    "IDE": "Programming development environment for code editing",
    "Figma": "Design tool for creating UI/UX prototypes and graphics",
    ...
}
test_data = [
    {"query": "Write a Python application to track the stock prices and generate a report.",
    "expected_agent": "Adam",
    "expected_tools": ["IDE", "Terminal"],},

    {"query": "Design a new logo for the company.",
    "expected_agent": "Eve",
    "expected_tools": ["Figma"]},
    ...
]

\end{lstlisting}

\subsection{Plugin Integration with Existing Systems - OpenWebUI (Integration)}

The final phase of the methodology focused on practical integration of the Zero Shot Router plugins with existing AI interfaces or platforms to demonstrate real world applicability.

The three plugins for OpenWebUI:

\begin{itemize}
    \item \textbf{Agent Router Plugin}: Routes arbitrary user queries to one of five specialised AI agents (Email, Code, Summariser, Chatbot, Sentiment Analysis) based on task descriptions.
    \item \textbf{Tool Router Plugin}: Selects the most appropriate system tool that has already been installed (e.g. web search, code interpreter, image generation) for each incoming user request.
    \item \textbf{Security Router Plugin}: An additional plugin was also created and tested to see if NLI could be used as a security guardrail.
\end{itemize}


\section{Fine Tuned Model}
\label{fine tuned model}
An important component of this project is evaluating whether specialised fine tuning of NLI models can outperform zero shot routing for our four core tasks. We therefore propose to train and compare four distinct fine tuned classifiers:

\begin{itemize}
    \item \textbf{Agent Selection Model}: Discriminates which agent (e.g. developer, designer, researcher) should handle a given prompt.
    \item \textbf{Tool Selection Model}: Identifies the most appropriate tools for a prompt (e.g. calculator, code executor, web browser).
    \item \textbf{Security Guardrail Model}: Flags adversarial or out of scope inputs (e.g. prompt injections, disallowed content).
    \item \textbf{Prompt Complexity Model}: Predicts the "difficulty" or resource demands of a prompt, aiding cost quality trade offs.
\end{itemize}

Each model will share the same base architecture (a BART-large-MNLI backbone) but receive task specific labelled data and classification heads. By fine tuning on dedicated datasets, we expect improved precision and recall over the zero shot NLI approach, particularly for nuanced or emerging patterns not well captured by generic NLI.

\subsection{Training Dataset}
\label{training dataset}

For the Training Dataset I will assemble and curate several datasets to support fine tuning:
\begin{itemize}
    \item \textbf{SoftAge-AI/prompt-eng\_dataset}: Provides a diverse set of real user prompts annotated for topic, complexity, tool usage, and agent role, supporting the Agent, Prompt Complexity, and Tool Selection models.
    \item \textbf{GuardrailsAI/restrict-to-topic}: Offers synthetic, topic restricted conversational examples, ideal for training the Security Guardrail model to detect off topic or forbidden content.
    \item \textbf{seankski/tool-parameters-v1-1-llama3-70B}: (or a similar parameter mapping dataset) Captures realistic mappings between prompts and tool invocation parameters for enriching the Tool Selection model’s training.
\end{itemize}

\subsection{Data Cleaning}
During the Data Cleaning for the \texttt{SoftAge-AI/prompt-eng\_dataset} dataset, I loaded the raw data from the Hub using the Hugging Face \texttt{datasets} library’s \texttt{load\_dataset} function, which seamlessly handles JSON and Parquet formats from both local and remote repositories. The dataset contained nested fields where each record comprised a JSON encoded conversation log and a JSON list of tool specifications. I implemented a custom parser to extract the first user utterance and the first tool’s description from each record, handling \texttt{JSONDecodeError}, missing keys, and empty lists robustly. Following extraction, I applied a filter step to remove any rows where parsing failed or returned empty strings. This reduced the raw corpus of approximately 551,285 examples to 441,028 valid training samples and 110,257 test samples, ensuring high quality inputs for downstream tokenisation and model training.



\subsubsection{4.4.2 Finetuning Process}
I selected the sequence classification variant of the BART large NLI model (\texttt{facebook/bart-large-mnli}) as our backbone, owing to its strong zero shot and fine tuning performance on sentence pair tasks. The model’s configuration was adapted to the 73 target classes as per the dataset and supplied with corresponding \texttt{id2label} and \texttt{label2id} mappings.

For optimisation, I used the \texttt{Trainer} API, which abstracts the training loop and automates gradient accumulation, checkpointing, and metric logging. Finetuning proceeded for one epoch, yielding stable training loss trajectories below 0.001 by step 1240. The final model was saved locally, and its mapping tables were exported to JSON for deployment.


\section{System Architecture}
\subsubsection{4.2.1 Overview of the Two Stage Routing Architecture}

The system architecture implements an elegant yet powerful two stage cascading router design that optimises both performance and functionality. While the system may appear complex at first glance, its fundamental structure follows a logical progression that systematically processes user prompts to deliver optimal results. The architecture consists of two primary components working in sequence: a \textbf{Model Router} followed by a \textbf{Tool Router}.

\subsubsection{4.2.2 Model Router: Intelligent Selection of Processing Engines}

The Model Router serves as the first decision layer in our system, determining which language model should process the incoming prompt. This critical routing decision is based on multiple factors including the prompt's domain, complexity, and specific requirements. The Model Router can operate in three increasingly sophisticated modes:

\textbf{Mode 1: Cost Performance Optimisation}  


\begin{figure}[H]
    \centering
    \includegraphics[width=0.5\textwidth]{figures/model-compx-router.drawio.png}
    \caption{Simple Complexity Router}
    \label{fig:simple_complexity_router}
\end{figure}


In its simplest configuration, the Model Router functions as a binary decision maker (as shown in this paper \cite{ong2025routellmlearningroutellms}), choosing between:

\begin{itemize}
    \item \textbf{Cost effective models}: Smaller, more efficient models with lower computational requirements, suitable for straightforward queries or scenarios with resource constraints.
    \item \textbf{High performance models}: More capable but resource intensive models reserved for complex reasoning, creative tasks, or specialized knowledge domains.
\end{itemize}

This mode optimises resource allocation by matching prompt complexity to the appropriate level of model capability, ensuring efficient use of computational resources while maintaining response quality.

\textbf{Mode 2: Domain Specialisation}  

\begin{figure}[H]
    \centering
    \includegraphics[width=0.5\textwidth]{figures/model-router.drawio.png}
    \caption{Domain Specialised Model Router}
    \label{fig:domain_specialised_model_router}
\end{figure}

In this more advanced configuration, the Model Router selects from an array of domain specialised models, each trained or fine tuned for excellence in particular knowledge areas. For example:

\begin{itemize}
    \item Code specialised models for programming tasks.
    \item Medical models for healthcare queries.
    \item Legal models for questions about law and regulations.
    \item Mathematical models for computational and quantitative problems.
\end{itemize}

This approach leverages the strengths of specialised training to deliver superior results in specific domains compared to general purpose models.

\textbf{Mode 3: Hybrid Routing with Cascading Filters}

\begin{figure}[H]
    \centering
    \includegraphics[width=0.5\textwidth]{figures/hybrid-agent-router.drawio.png}
    \caption{Hybrid Model Router}
    \label{fig:hybrid_model_router}
\end{figure}

The most sophisticated implementation combines the previous approaches into a comprehensive routing strategy. In this configuration, the system:

\begin{enumerate}
    \item First evaluates the prompt against domain categories to identify specialised knowledge requirements.
    \item Then assesses complexity factors to determine the appropriate performance tier within that domain.
    \item Finally selects the optimal model that balances domain expertise with appropriate computational resources.
\end{enumerate}

This hybrid approach gets the best of both worlds by ensuring prompts receive both domain appropriate handling and suitable computational resources.

\subsubsection{Tool Router: Extending Model Capabilities Through External Functions}

After the appropriate model has been selected, the Tool Router provides the second layer of intelligence by determining which external tools or knowledge sources should supplement the model's capabilities. The Tool Router:

\begin{enumerate}
    \item Analyses the prompt to identify specific functional requirements that might benefit from specialised tools.
    \item Selects appropriate external utilities from its available toolset.
    \item Orchestrates the interaction between the selected model and tools via standardised function calling interfaces.
\end{enumerate}

\subsubsection{Integration and Information Flow}

The complete system operates as a seamless processing pipeline:

\begin{enumerate}
    \item A user prompt enters the system.
    \item The Model Router evaluates the prompt's domain and complexity requirements.
    \item Based on this evaluation, the appropriate model is selected.
    \item The selected model begins processing the prompt.
    \item Concurrently, the Tool Router identifies any external tools required.
    \item If tools are needed, the model interacts with them via function calling.
    \item The integrated results from both model processing and tool outputs are combined.
    \item A comprehensive response is returned to the user.
\end{enumerate}

This architecture enables sophisticated query handling that dynamically adapts to varying prompt requirements while maintaining system efficiency. By separating model selection from tool selection, the system achieves a high degree of flexibility and extensibility, allowing for independent optimization of each component.

In this script, we set up a simple command line interface that allows users to input prompts and receive routing results. The script uses the \texttt{AgentRouter}, \texttt{ToolRouter}, and \texttt{Router} classes from the \texttt{llm\_routers} library to route the input prompt to the appropriate agents, tools, and complexity levels. The results are printed in a user friendly format.

For the demo script, we also added a signal handler to gracefully shut down the routers when the user interrupts the script (e.g., by pressing Ctrl+C). This ensures that any resources used by the routers are properly released.

\subsection{Plugin Integration with Existing Systems}

To demonstrate the practical applicability of the routing system, we integrated the \texttt{llm\_routers} library with an existing AI interface. The integration process involved creating plugins for OpenWebUI, a popular open source web based interface for interacting with large language models.

OpenWebUI allows users to interact with various AI models and tools through a web interface. It is a platform that supports custom plugins, via the admin panel, enabling developers to extend its functionality by adding new \texttt{functions} and \texttt{tools}. Using the functions system, we can create plugins that route user queries to the appropriate agents and tools based on the routing decisions made by the \texttt{llm\_routers} library.

Creating a plugin for OpenWebUI involves reading the OpenWebUI documentation from \url{https://docs.openwebui.com/pipelines/pipes/} and following the guidelines for plugin development.

\subsubsection{Model Router Plugin}

My first obstacle was that the OpenWebUI API to get the list of available models was not working. I had to manually create a list of models and their descriptions. The API, however, was working for the tools, which updated the list of available tools if new tools were added or removed.

For the Model Router plugin, to pass this obstacle I created a dictionary of available models and their descriptions. The plugin required a \texttt{pipe} class with a \texttt{pipe} method that runs after the user input is received. The \texttt{pipe} method then calls the \texttt{AgentRouter} classes from the \texttt{llm\_routers} library to route the user query to the appropriate agents. The plugin currently returns a debug like message with the chosen agent and some other information without actually running the agent or inferring any agent. This was done to keep the plugin simple and easy to understand just the core functionality of the router. Although the plugin can be extended to run the agent in the future.

\begin{figure}[H]
    \centering
    \includegraphics[width=0.9\textwidth]{figures/owui-agent-demo-0.png}
    \caption{Example of the Model Router using the OpenWebUI API Where it has successfully routed the user to an Email assistant.}
    \label{fig:model_router_plugin_demo_0}
\end{figure}

\begin{figure}[H]
    \centering
    \includegraphics[width=0.9\textwidth]{figures/owui-agent-demo-1.png}
    \caption{Example of the Model Router using the OpenWebUI API Where it has successfully routed the user input to a Codeing assistant.}
    \label{fig:model_router_plugin_demo_1}
\end{figure}

\begin{figure}[H]
    \centering
    \includegraphics[width=0.9\textwidth]{figures/owui-agent-demo-2.png}
    \caption{Example of the Model Router using the OpenWebUI API Where it has successfully routed the user input to an chatbot agent.}
    \label{fig:model_router_plugin_demo_2}
\end{figure}



\subsubsection{Tool Router Plugin}

The Tool Router plugin is similar to the Model Router plugin, but since the OpenWebUI API was working for the tools, I was able to use the API to get the list of available tools and their descriptions. The plugin first gets the list of available tools from the OpenWebUI API and initialises the \texttt{ToolRouter} class.

Since the Tool Router plugin is more complex than the Model Router plugin, this plugin used a \texttt{filter} method from OpenWebUI. Where the \texttt{filter} method allows the plugin to modify the user input before and after the inference. The \texttt{inlet} method is called before the user input is sent to the model, and the \texttt{outlet} method is called after the model output is received. This allows the plugin to modify the user input and model output before and after the inference.

Since we need to select the tool before the user input is sent to the model, we used the \texttt{inlet} method to route the user input to the appropriate tools. Here we used the \texttt{ToolRouter} class to route the user input to the appropriate tools.

Within this plugin, I also got the chance to work with other APIs that OpenWebUI provides. For example, the \texttt{EventEmitter} API allows the plugin to show a message in the OpenWebUI interface. This was used to show the user which tools were selected for the user input.

\begin{figure}[H]
    \centering
    \includegraphics[width=0.9\textwidth]{figures/owui-tool-demo-0.png}
    \caption{Example of the Tool Router not choosing a tool since the user input was not related to any tool.}
    \label{fig:tool_router_plugin_demo_0}
\end{figure}

\begin{figure}[H]
    \centering
    \includegraphics[width=0.9\textwidth]{figures/owui-tool-demo-1.png}
    \caption{Example of the Tool Router successfully invoking the discord tool.}
    \label{fig:tool_router_plugin_demo_1}
\end{figure}

\begin{figure}[H]
    \centering
    \includegraphics[width=0.9\textwidth]{figures/owui-tool-demo-2.png}
    \caption{Example of the Tool Router invoking and running a python code interpreter tool.}
    \label{fig:tool_router_plugin_demo_2}
\end{figure}


\subsubsection{Security Router Plugin}

Similar to the Model Router, this too uses the \texttt{Pipe} class and the \texttt{pipe} method to route the user input to the appropriate security guardrail.

Since the security guardrail is a simple text classification task, we used the \texttt{Router} class from the \texttt{llm\_routers} library to categorise the user input into either \texttt{prompt injection}, \texttt{Data Leakage}, \texttt{Model Evasion}, \texttt{Adversarial Examples}, \texttt{Malicious Code}, or \texttt{Malicious Query}. The plugin then returns a debug like message with the selected type of attack and the confidence score. Since this is a rather complex task, the plugin is not very accurate and is not recommended for use. Although the plugin can be extended to use a more complex model or by using a fine tuned model as described in the previous section.
